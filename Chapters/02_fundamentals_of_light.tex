\chapter{Fundamentals of Light}
\label{ch:fundamentals_of_light}


\section{History of Optics}

The systematic story of light dates back at least to the ancient Greek Philosophers \cite{Vohnsen_2004}, who sought to understand the nature of light and its role in visual perception. Three school of thought originated from those studies: (1) The Pythagoreans imagined that something emitted by the eye interacted with the object in front to create an image. (2) Democritus hypothesized exactly the opposite; that light is something emitted by a object that carries information about the object's shape and color, which interacts with the human eye. Followers of (3)  Empedocles believed light to be a combination of the previous two ideas. These philosophical investigation turned quantitative  with Euclid's introduction of geometric optics in 300 B.C.\cite{Abetti1971}, specifically with his discovery of the law of reflection. The ensuing two millennia of observations and theorizing about the nature of light were distilled by the end of the seventeenth century into two competing theories: the corpuscular theory of light proposed by Newton in his book Optics (1704) and the wave theory developed by Huygens, XXXXX [Find ref] and others. Another two centuries passed before this dichotomy was resolved with the quantitative theory of light. The present thesis treats light as a wave and adopts the analytical description of wave optics pioneered by Fresnel in 1819. Fresnel paved the way for a solid wave theory of light by conducting several experiments which confirmed that light propagates as a sum of Huygens wave. 

\begin{figure}
\begin{subfigure}{.3\textwidth}
  \centering
  \includegraphics[width=.6\linewidth]{pythagoras}
  \label{fig:pythagoras}
\end{subfigure}%
\begin{subfigure}{.3\textwidth}
  \centering
  \includegraphics[width=.6\linewidth]{democritus}
  \label{fig:democritus}
\end{subfigure}
\begin{subfigure}{.3\textwidth}
  \centering
  \includegraphics[width=.9\linewidth]{empedocles}
  \label{fig:empedocles}
\end{subfigure}%
\caption{Three of the most popular school of thought about the properties of light in the age of Greek philosophers: Pythagoras (left) conjectured that eyes interacted with an object by emitting something. Source: FixQuotes.com \cite{pythagoras_pic}. Democritus's (middle) view was the opposite; the eyes receive information about the object through something which is emitted by the item. Source: Atomic Model Timeline \cite{democritus_pic}. And Empedocles (right) believed both the eyes and the object emits information in some form in order for it to be perceived by humans. Source: History-biography.com \cite{empedocles_pic}. XXXX [Change image ref]}
\label{fig:early_hypothesis}
\end{figure}

\section{Light is Electromagnetic Waves}
Fifty years after Fresnel first formulated wave optics, Maxwell provided theoretical foundation for the wave theory of light through his theory for electromagnetism first reported in 1861. Maxwell synthesized the preceding century

 of the now famous Maxwell's Equations which can be found in his 1861 paper \cite{ClerkMaxwell_1861}. These equations :
\begin{subequations}
\label{eq:maxwelleqs}
\begin{equation}
\label{eq:maxwellcoulomb}
\nabla\cdot \vec{E}                \quad  = \quad \frac{\rho}{\epsilon _0} \quad,                          \quad \text{(Coulomb)}
\end{equation}
\begin{equation}
\label{eq:maxwellgauss}
\nabla\cdot \vec{B}                \quad  = \quad 0\quad,                          \quad \text{(Gauss)}   \\[5pt]
\end{equation}
\begin{equation}
\label{eq:maxwellamp}
\quad \nabla\times\vec{E}   = -\frac{\partial\vec{B}}{\partial t} \quad,   \quad \text{(Faraday)}   \\[5pt]
\end{equation}
\begin{equation}
\label{eq:maxwellfaraday}
\quad \nabla\times\vec{B}  = \mu _0 \vec{J} + \mu _0 \epsilon _0 \frac{\partial \vec{E}}{\partial t} \quad,    \quad \text{(Ampère)} \\[5pt]
\end{equation}
\end{subequations}
These four equations summarize how the time varying fields $\electricfield$ and $\magneticfield$ are related to electric charge density $\rho (\vec{r},t)$ and the electric current density $\vec{J}(\vec{r},t)$. Equation~\eqref{eq:maxwellcoulomb}, also known as the Gauss' Law, tells us how charge density creates electric field. Equation~\eqref{eq:maxwellgauss} states the experimental fact that magnetic monopoles do not seem to exist, which requires that divergence of the magnetic field $\magneticfield$ must vanish. Equation~\eqref{eq:maxwellfaraday} also known as Faraday's Law that shows current can be induced with in a loop or wire if a changing magnetic field slices through it. The fourth of the Maxwell's Equations describes how electric current give rise to magnetic fields.This typically is credited to Ampère's although Maxwell himself incorporated the coupling between $\electricfield$ and $\magneticfield$ to obtain a symmetric set of equations. In vacuum $\rho(\vec{r},t) = 0$ and $\vec{J}(\vec{r},t) = 0$ so that Maxwell's Equations simplify to :
\begin{subequations}
\begin{equation}
\label{eq:maxwell_E}
\nabla\cdot \vec{E}  = 0 \quad, \quad  \nabla\times\vec{E}   = -\frac{\partial\vec{B}}{\partial t}
\end{equation}
\begin{equation}
\label{eq:maxwell_M}
\nabla\cdot \vec{B}  = 0 \quad, \quad \nabla\times\vec{B}  =  \mu _0 \epsilon _0 \frac{\partial \vec{E}}{\partial t}
\end{equation}
\end{subequations}
Taking the curl of Eq.~\eqref{eq:maxwell_E} and Eq.~\eqref{eq:maxwell_M} we obtain the wave equations for the electric and magnetic fields:
\begin{subequations}
\begin{equation}
\label{eq:waveeq_E}
\left( \nabla ^2 + \frac{1}{c^2}\frac{\partial ^2}{\partial t^2}\right)\vec{E} = 0 \quad ,
\end{equation}
\begin{equation}
\label{eq:waveeq_B}
\left( \nabla ^2 + \frac{1}{c^2}\frac{\partial ^2}{\partial t^2}\right)\vec{B} = 0 \quad ,
\end{equation}
\end{subequations}
where the constant $c$ is defined as:
\begin{equation}
\label{eq:speed_of_light}
c = \frac{1}{\sqrt{\mu _0 \epsilon _0}} \quad ,
\end{equation}
which is same as the speed of light. In $\mathrm{1865}$ Maxwell noted that $c$ is consistent with the speed of light as suggested in this thesis that light is an electromagnetic wave \cite{ClerkMaxwell_1865}. The same wave equations also predicted the existence of electromagnetic wave with frequencies outside the range of visual perception. Later between $1886$ and $1889$ Hertz conducted several experiments to prove Maxwell's prediction. In his seminal paper: ``On Electromagnetic Effects Produced by Electrical Disturbances in Insulators'', Hertz showed that electromagnetic waves traveling at the speed of light \cite{dagostino1975}.

\section{Solution of Wave Equations}
\label{chaptersec:Solution of Wave Equations}
The general solutions to Eq.~\eqref{eq:waveeq_E} and Eq.~\eqref{eq:waveeq_B},
\begin{subequations}
\begin{equation}
\label{eq:Electridfield}
\vec{E}(\vec{r},t) = E_0 \exp \left(i\vec{k}\cdot \vec{r} - i\omega t\right) \vec{\pol} \quad \mathrm{and}
\end{equation}
\begin{equation}
\label{eq:Magneticfield}
\vec{B}(\vec{r},t) = \frac{E_0}{c} \exp \left(i\vec{k}\cdot \vec{r} - i\omega t\right) \hat{k}\times \vec{\pol}
\end{equation}
\end{subequations}
represent monochromatic plane waves \cite{jackson_classical_1999}, where $\vec{k}$ is the wave vector that tells us the direction of propagation of the wave and $\vec{\pol}$ is the axis of polarization. The wave number $k = |\vec{k}|$ is connected to $c$, the speed of light, through the dispersion relation: $k = \omega / c$ and the wavelength can be calculated from $\lambda = 2\pi  / k$. Both electric (Eq.~\eqref{eq:Electridfield}) and magnetic (Eq.~\eqref{eq:Magneticfield}) fields are represented as a complex-valued functions because it is convenient for calculation. $\real \electricfield$ and $\real \magneticfield$ are the real part of $\electricfield$, and $\magneticfield$ respectively and they correspond to the actual electric and magnetic field. The work described in this thesis is carried out with linearly polarized light, and we all adopt the convention $\hat{\epsilon} = \hat{x}$, due to the nature of our experimental setup, from here on.

\section{Experimental Parameters}

In Cartesian coordinates the light field can be described by six complex valued functions, two for each coordinate axis. The electric field of a monochromatic beam of light in Cartesian coordinate can be described as:
\begin{equation}
\label{eq:complexfield}
\vec{E}(\vec{r},t) = {\displaystyle\sum_{j=1}^{3}}E_j\left(\vec{r}\right)\exp \left(-i\omega t\right)\vec{\polj}(\vec{r})
\end{equation}
where $E_j(\vec{r})$ is the complex scalar field and $\omega$ is the frequency of the light. As described in \ref{chaptersec:Solution of Wave Equations} we only work with linearly polarized light. $E_j(\vec{r})$ is a solution of the Helmholtz Equation \cite{goodmanfourier}:
\begin{equation}
\left(\nabla ^2 + k^2\right) E = 0
\end{equation}
and in the paraxial limit of the Helmholtz Equation $E_j(\vec{r})$ can be expressed as:
\begin{equation}
\label{eq:E_amp_phase}
E_j \left(\vec{r}\right) = u_j\left(\vec{r}\right) e^{i\phi _j (\vec{r})}\quad ,
\end{equation}
where $u_j(\vec{r})$ is the amplitude and $\phi (\vec{r})$ is the phase of the scalar field.
  


\section{Diffraction of Light}

The first quantitative study of the deviation of light from its rectilinear propagation \cite{hechtoptics}, is a phenomenon known as ``diffraction'', was reported by Francesco Grimaldi \cite{bornwolf} in 1665. A great victory of the wave theory of light is its ability to account naturally for diffraction. This description underlies the approach adopted in this thesis to describe the holographic video microscopy and holographical optical trapping. To understand digital holographic microscopy \cite{Lee:07} it is essential to understand the limitations imposed by diffraction.

\subsection{Rayleigh-Sommerfeld Diffraction Theory}
According to Huygens principle every point on a wavefront of a wave can be considered to be a secondary source which creates a spherical wavefront. Fresnel proposed that such secondary wavefronts recreate the wavefronts of the primary wave by interfering with each other, is known as the ``Huygens - Fresnel Principle''. Using the geometry presented in the Fig.\ref{fig:huygens_fresnel} , the electric field at point ``$\mathbf{P}$'' due to the secondary wave generated a small area $dS$ at point ``$\mathbf{Q}$'' can be written as:

\begin{figure}[t!]
  \centering
  \includegraphics[width=0.7\textwidth]{huygens_fresnel_schematic}
  \caption{According to Huygens-Fresnel principle a sample point $\mathbf{Q}$ is considered as a secondary source which emits a spherical wavefront. The electric field at point $\mathbf{P}$ is a superposition of all secondary wavefronts created on the surface of the parent wavefront with center at $\mathbf{O}$.}
  \label{fig:huygens_fresnel}
\end{figure}


\begin{equation}
\label{eq:huygen_fresnel}
dE(\vec{P}) = K(\chi) \frac{u_0 e^{ik \cdot r_0}}{r_0}\frac{e^{i\vec{k}\cdot\vec{\ell}}}{\ell} d\vec{S} \quad ,
\end{equation}
where $r_0$ is the radius of the parent spherical wavefront originated from point ``$\mathbf{O}$'' and $\ell$ is the distance between point ``$\mathbf{Q}$'' and ``$\mathbf{P}$''. $K(\chi)$ is the inclination factor which is maximum ($1$) when the propagation direction ``\textbf{OQ}'' aligns with ``\textbf{OP}''. Therefore the total field at ``\textbf{P}'' will be:
\begin{equation}
\label{eq:E_P}
E(\vec{P}) =  \frac{u_0 e^{ikr_0}}{r_0} \int \int _{S} \frac{e^{i\vec{k}\vec{s}}}{s}  K(\chi) d\vec{S} \quad .
\end{equation}

Kirchoff \cite{kirchoff1883} showed that Huygens-Fresnel principle is an approximation of the now well known ``Fresnel-Kirchoff Diffraction Formula'':
\begin{equation}
\label{eq:fresnel_kirchoff}
E(\vec{P}) = -\frac{iu_0}{2\lambda}\int \int _S \frac{e^{ik(r+s)}}{rs}\left[\cos (\vec{n},\vec{r}) - \cos (\vec{n},\vec{s})\right]d\vec{S} \quad ,
\end{equation}
which describes the electric field at \textbf{P} due to diffraction of light originated at $\mathbf{P_0}$ through a planar aperture as depicted in Fig.~\ref{fig:kirchoff_diffraction}.
\begin{figure}[t!]
  \centering
  \includegraphics[width=0.7\textwidth]{kirchoff_diffraction}
  \caption{Fresnel-Kirchoff Diffraction formula describes the electric field at point $\mathbf{P}$ due to a point source placed at $\mathbf{P_0}$ and an aperture placed in between.}
  \label{fig:kirchoff_diffraction}
\end{figure}
The boundary conditions imposed on both the field and its normal derivative in order to obtain the Fresnel-Kirchhoff diffraction formula are known to be mathematically inconsistent \cite{Lucke_2006, Heurtley:73, Sommerfeld:1954:O}.  The diffraction formula shows strong deviation from the physical solution when the observation point is close to the diffracting screen. It also yields an incorrect intensity pattern for Poisson's spot created by diffraction from an annular aperture. Sommerfeld corrected these inconsistencies by choosing an alternative Green's function and removing the boundary condition on the normal derivative of the field. His solution:
\begin{equation}
\label{eq:rayleigh_sommerfeld}
E(\vec{P}) = -\frac{iu_0}{\lambda}\int \int _S \frac{e^{i\vec{k}(\vec{r}+\vec{s})}}{rs} \cos (\vec{n},\vec{s}) d\vec{S} \quad ,
\end{equation}
is known as the ``Rayleigh-Sommerfeld Diffraction Formula''.

\subsection{Fresnel and Fraunhofer Diffraction}

Equation~\eqref{eq:rayleigh_sommerfeld} can be rewritten in terms of the field in the aperture as:
\begin{equation}
\label{eq:rayleigh_updated}
E(\vec{P}) = \frac{1}{i\lambda}\int \int _S E(\vec{Q}) \frac{e^{i\vec{k}\vec{s}}}{s} \cos (\theta) d\vec{S} \quad ,
\end{equation}
where $E(\vec{Q})$ is the field at \textbf{Q} on the aperture and $\theta$ is $\cos (\vec{n},\vec{s})$, the angle between the normal to the aperture and the vector $\vec{s}$. Assuming Cartesian coordinates to these points,
\begin{subequations}
\begin{equation}
P_0 \equiv \left(x_0, y_0, z_0\right) \quad ,
\end{equation}
\begin{equation}
P \equiv \left(x, y, z\right) \quad ,
\end{equation}
\begin{equation}
Q \equiv \left( \xi , \eta \right) 
\end{equation}
\end{subequations}
yields $\cos \theta = \frac{z}{s}$ and the Eq.~\eqref{eq:rayleigh_updated} simplifies to:
\begin{equation}
\label{eq:rayleigh_simple}
E\left( x,y\right) = \frac{z}{i\lambda}\int \int _S E(\xi,\eta) \frac{e^{i\vec{k}\vec{s}}}{s^2} d\xi d\eta \quad ,
\end{equation}
where:
\begin{equation}
s = \sqrt{z^2 + \left( x - \xi \right) ^2 + \left( y-\eta \right) ^2} \quad .
\end{equation}
The Fresnel approximation:
\begin{equation}
\label{eq:fresnel_approx}
s \approx z\left[ 1 + \frac{1}{2}\left(\frac{x-\xi}{z}\right)^2 + \frac{1}{2}\left(\frac{y-\eta}{z}\right)^2\right]
\end{equation}
further simplifies Eq.~\eqref{eq:rayleigh_simple} and we get:
\begin{equation}
\label{eq:fresnel_diffraction}
\begin{split}
E\left( x,y\right) = \frac{e^{ikz}}{i\lambda z}e^{i\frac{k}{2z}(x^2+y^2)}\int \int _{S} & E\left( \xi , \eta \right) e^{i\frac{k}{2z}(\xi^2+\eta ^2)} \\
& \times e^{-i\frac{k}{2z}(x\xi+y\eta)} d\xi d\eta \quad ,
\end{split}
\end{equation}
which is valid in the near field of the aperture. In the far-field the Fraunhofer approximation \cite{goodmanfourier}:
\begin{equation}
\label{eq:fraunhofer_approx}
z \gg \frac{k\left( \xi ^2 + \eta ^2\right)}{2}
\end{equation}
simplifies Eq.~\eqref{eq:fresnel_diffraction} even further to:
\begin{equation}
\label{eq:fraunhofer_diffraction}
E\left( x,y\right) = \frac{e^{ikz}}{i\lambda z}e^{i\frac{k}{2z}(x^2+y^2)}\int \int _{S}  E \left( \xi , \eta \right) e^{-i\frac{k}{2z}(x\xi+y\eta )} d\xi d\eta \quad ,
\end{equation}
which is same as the Fourier transform of the field in the aperture.


\section{Optical Forces}

\setlength{\epigraphwidth}{0.8\textwidth}
\epigraph{``It is probable that a much greater energy of radiation might be obtained by means of concentrated rays from an electric lamp. Such rays falling on a thin metallic disc, delicately suspended in a vacuum, might perhaps produce an observable mechanical effect''}{\textit{J C Maxwell, 1873}}

\begin{figure}
\begin{subfigure}{.4\textwidth}
  \centering
  \includegraphics[width=.5\linewidth]{kepler}
  \label{fig:kepler}
\end{subfigure}%
\begin{subfigure}{.4\textwidth}
  \centering
  \includegraphics[width=.9\linewidth]{halley}
  \label{fig:halley}
\end{subfigure}
\caption{German mathematician and astronomer Johannes Kepler (left) conjectured the radiation pressure of sunlight could explain why comet tails always point away from the sun including the Halley's comet's (right) tail. Source: Wikipedia \cite{kepler_pic, halley_comet}.}
\label{fig:kepler_halley}
\end{figure}

The first proposal that light might exert forces was made by Kepler in $1619$, when he suggested the tail of Halley's comet might be created by the radiation pressure of sunlight. Two and a half centuries later Maxwell used his theory of electromagnetism to demonstrate that electromagnetic waves carry momentum and that this momentum can be transferred to illuminated objects as radiation pressure while it interacts with the object. This momentum transfer can happen either via reflection/scattering or absorption. 

In the past fifty years there has been a significant increase in the interest of understanding forces exerted by an electromagnetic wave interacting with small objects. Arthur Ashkin first pointed out in 1970 that optical forces could provide convenient ways to control the dynamics of small objects and that this would have major applications inatomic physics, biology and nonlinear physics. Light electric field and magnetic field exert forces on small neutral objects by inducing time varying chharge multipoles in them and then exerting forces and torques on the induced multipoles [Find ref] \cite{jackson_classical_1999,Chaumet:00, gordon1973, Dungey:91}. For particles that are much smaller than the wavelength of light the Lorentz force is dominated by dipole contributions. The induced dipole moment experience a force in gradients of the electric field  $(\vec{p}\cdot \vec{\nabla})\vec{E}$. The time-varying dipole moment acts like a current that couples to the magnetic field, $\vec{\dot{p}}\times \vec{B}$. The resulting dipole order force has the time averaged form \cite{gordon1973}

\begin{equation}
\label{eq:Lorentz force}
\vec{F_e}  = \frac{1}{2} \Re \lbrace (\vec{p}\cdot \vec{\nabla})\vec{E}^\star + \frac{1}{c} \dot{\vec{p}}\times \vec{B}^\star \rbrace 
\end{equation}
The induced dipole is proportional to the local electric field, $\hat{\vec{p}} = \alpha _{e}\vec{\hat{E}}$, where $\alpha _{e}$ is the particle's electric dipole polarizability. In 2000, Chaumet and Vesperinas \cite{Chaumet:00} ahowed that Eq.~\eqref{eq:Lorentz force} could be rewritten in the compact and evocative form:
\begin{equation}
\label{eq:chaumet force}
\vec{F_e} = \frac{1}{2} \Re \lbrace \alpha _{e} E_j \partial _i E^\star _{j} \rbrace \quad ,
\end{equation}
This form is useful for developing a practical framework for controlling optical forces. The optical force is parametrized by the object's polarizability, which depends on its size,shape and chemical composition. $\alpha _e$ is complex valued and can written as $\alpha _e = \alpha^{\prime}_e + i \alpha ^{\prime \prime}_{e}$, where $\alpha ^{\prime}_{e}$ and $\alpha ^{\prime \prime}_{e}$ are the real and imaginary part \cite{jackson_classical_1999} respectively. The imaginary part of the polarizability accounts for absorption and radiative losses. For the special case of a sphere of radius $a_p$ and refractive index $n_p$ in a medium with refractive index $n_m$ the electric polarizability is given by the Clausius-Mosotti-Draine relationship \cite{draine1993}:

\begin{equation}
\label{eq:clausius mosotti}
\alpha _e = \frac{4\pi \epsilon _0 n^2_m K a^3_p}{1-i\frac{2}{3}K k^3 a^3_p} \quad \textsf{, where }\quad K = \frac{n^2_p - n^2_m}{n^2_p + 2n^2_m}.
\end{equation}
Here $\epsilon _{0}$ is the permittivity of space and $k$ is the wave number of the light. Equation.~\eqref{eq:chaumet force} and Eq.~\eqref{eq:clausius mosotti} specify what force can be expected from a specified electric light field. They are less useful for designing waves to exert derived forces. A more interpretable expression can be obtained by replacing the electric fields in Eq.~\eqref{eq:Lorentz force} with Eq.~\eqref{eq:E_amp_phase} which yields:

\begin{equation}
\label{eq:Electric Lorentz force}
\vec{F_e}(\vec{r}) = \frac{\omega ^2}{4}\alpha ^{\prime} _{e} \sum _{j=0}^{2}\nabla u_{j}^{2}(\vec{r}) + \frac{\omega ^2}{2} \alpha ^{\prime \prime}_{e} \sum _{j=0}^{2} \nabla u_{j}^{2}\phi _{j}(\vec{r})
\end{equation}

The first term in Eq.~\eqref{eq:Electric Lorentz force} is proportional to the gradient of the light intensity. It is manifestly conservative and tends to draw dielectric particles ($\alpha _{e}^{\prime}>0$) toward intensity maxima. This intensity gradient force allows focused beams if light to trap small objects in three dimensions, thereby acting as ``optical tweezers'' \cite{Ashkin:86}. The second term in Eq.~\eqref{eq:Electric Lorentz force} is directed by gradients of the phase and describes a non-conservative force that identify with the radiative pressure.

