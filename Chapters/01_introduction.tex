\chapter{Introduction}
\label{ch:intro}

%    Move 1 establish your territory (say what the topic is about)
%    Move 2 establish a niche (show why there needs to be further research on your topic)
%    Move 3 introduce the current research (make hypotheses; state the research questions)


%    state the general topic and give some background
%    provide a review of the literature related to the topic
%    define the terms and scope of the topic
%    outline the current situation
%    evaluate the current situation (advantages/ disadvantages) and identify the gap
%    identify the importance of the proposed research
%    state the research problem/ questions
%    state the research aims and/or research objectives
%    state the hypotheses
%    outline the order of information in the thesis
%    outline the methodology

% https://www.scribbr.com/theses-examples/examples-dissertation-phd-theses/

%\section{Topological Modes of Light}
%
%\begin{itemize}
%	\item Introduce angular momentum (See: Angular Momentum of Light)
%	\item What do  we mean by topological modes?
%		\begin{itemize}
%			\item Why are they important?
%			\item Wavefronts with topological defects
%			\item Why not consider non-topological modes?
%		\end{itemize}
%\end{itemize}
%
%Give an idea of why you chose this project

\section{Why is this Problem Relevant?}

From Maxwell's theory it is known that an electromagnetic wave or optical field carries both momentum and energy. Momentum can be further decomposed into linear and angular momentum, where angular momentum has two components; the spatial structure gives rise to orbital angular momentum \cite{Yao2011} and the spin angular momentum is associated with its polarization. Therefore any interaction between the optical field and matter will give rise to momentum transfer hence a force acting on the interacting body. In case of an optical field that is described by a simple planar wave the force acts as a radiation pressure that pushes any object downstream, away from the source. Multiple plane waves can be superposed in a way that generates attractive force which will pull an object upstream towards the source. Such superposition of planar waves was first discovered by Arthur Ashkin and his co-workers in Bell lab during $1986$ \cite{beth1936}, which is known as optical tweezers. Since this discovery optical manipulation has become an active topic of research, where complex optical fields are generated using various means which serve several purposes ranging from exerting non-invasive forces on biological samples to mimicking quantum mechanical systems to study its properties.

An interest in deep space exploration has urged the need to answer the question: what kind of force can be generated using light field to pull an object from a distance? Emphasis on faster optical communication systems has lead research in understanding the physical limitation of the propagation distance of an optical field. In this thesis we explore options to make ``long-range tractor beams'' into a reality and understand its properties.

Understanding the local properties of a beam of light is the stepping stone for studying the evolution of an optical field and the forces generated by it. The energy flux of a plane wave can be calculated using the Poynting vector, which also tells us the direction of the radiation pressure. Optical trapping of dielectric particle requires strong intensity gradient which is independent of the direction of propagation of light. In order for successful trapping the intensity gradient force needs to be higher than the radiation force exerted by the field of light. In such an optical trapping system the trapped particle feels an attractive force towards the point of intensity maximum.

The limitation of a point optical trap created using strongly focused laser beam with a high intensity gradient is its range. The distance over which the intensity gradient force can overcome the radiation force goes only up to the order of $~1\mathrm{\mu m}$. Such a short range gives way to the problem of finding a mode of light which can achieve long range optical manipulation. The ideal optical mode  needs to be propagation invariant for long range manipulation and create a net negative force field through momentum transfer when it interacts with an object. Previously, exotic modes were created which are spiraling around their axis were shown to act as a tractor beam \cite{Lee_2010}. But such tractor beams were realized over $\sim 100 \mathrm{\mu m}$ only. Here we present intermediate plane holography a new experimental technique for creating such tractor beams with increased power efficiency and larger propagation invariant range.

Due to the homology of paraxial wave equation with Schrodinger's equation the spatial structure of a propagating beam of light is analogous to the temporal evolution of a quantum mechanical wave packet. Therefore we can use optics to understand complex quantum mechanical states. A spiraling beam of light is a superposition of multiple Bessel beam with non-zero angular momentum. A similar wave function arises inside a infinite cylindrical wall in quantum mechanics. In this thesis we have studied the propagation of a spiral beam to understand the evolution of a Bessel like quantum state under a force free condition, which is a class of accelerating beam in the same sense as reported by Berry and Balazs \cite{Berry1979}.

%First experimental realization of optical angular momentum was done by Beth \cite{beth1936}. 

%\section{Distilling Important Physical Processes}
%\section{Applying and Extending Holographic Video Micrscopy}



\section{Organization}

This thesis is organized in the following manner:

Chapter \ref{ch:fundamentals_of_light} starts with a brief historical motivation behind understanding the properties of light followed by the basic formalism for describing it in terms of electric and magnetic field. We introduce the generalized version of Maxwell's equations and arrive at the general solutions of electric and magnetic field which are the solution of wave equations. In order to maintain consistency we introduce the parameters used in our experiments and use them to describe the propagation of electric field. 

Using the fundamental description of light, in Chapter \ref{ch:topological_modes}, we describe different types of topological modes of light generated using digital holographic technique.

Chapter \ref{ch:intermediate} introduces a new technique called ``Intermediate Plane Holography (IPH)''  for structuring laser beams with computer-generated holograms. IPH can dramatically improve both diffraction efficiency and mode purity. We illustrate these capabilities by projecting Bessel beams, which constitute the natural basis for propagation-invariant modes \cite{durnin87,durnin87a}.

Chapter \ref{ch:accelerating} presents a new class of accelerating beam in two dimensions. Here we demonstrate that a shape-preserving wave packets that rotate at constant angular speed around the center of the box follows Ehrenfest's theorem. The apparent violation of Ehrenfest’s theorem is resolved by considering the force exerted on the particle’s wave packet by the enclosing wall.

Finally, Chapter 6 shows possible pathways for extending research in this field and its applications. (Yet to be done!!)
%\section{Outline of Dissertation}
%\section{Thesis Overview}

%This thesis yada yada yada
