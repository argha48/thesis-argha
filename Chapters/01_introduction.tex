\chapter{Introduction}
\label{ch:intro}

%    Move 1 establish your territory (say what the topic is about)
%    Move 2 establish a niche (show why there needs to be further research on your topic)
%    Move 3 introduce the current research (make hypotheses; state the research questions)


\section{Why does long-range holography matter?}

From Maxwell's theory it is known that an electromagnetic wave or optical field carries both momentum and energy. Momentum can be further decomposed into linear and angular momentum, where angular momentum has two components: the spatial structure gives rise to orbital angular momentum \cite{Yao2011} and its polarization is associated with the spin angular momentum. Therefore any interaction between the optical field and matter will give rise to momentum transfer and thus forces, and torques acting on the illuminated body. A simple plane electromagnetic wave exerts uniform radiation pressure that pushes any illuminated objects downstream, away from the source. Remarkably multiple plane waves can be superposed in a way that generates attractive forces that pull an object upstream towards the source. Such a superposition of plane waves was first discovered by Arthur Ashkin and his co-workers at Bell Labs in $1986$ \cite{beth1936}, and is known as optical tweezers. Since this discovery, optical manipulation has become an active topic of research, where complex optical fields are generated using various means and serve purposes ranging from non-invasive manipulation of biological samples to fundamental studies in statistical physics \cite{Volpe:09,Sanchez2019,Leibler1994}.


An interest in deep space exploration has inspired the desire to answer the question: what kind of useful force can light exert on an object at a great distance? Similarly the commercial need for faster optical communication systems has inspired research in to the practical limits on the propagation distance of structured optical fields. In this thesis we explore options to make ``long-range tractor beams'' into a reality and understand their properties.


Understanding the local properties of a beam of light is the starting point for studying the evolution of an optical field and the forces generated by it. The energy flux of a plane wave can be calculated using the Poynting vector, which also provide insights to the direction of the radiation pressure. Radiation pressure can be counteracted by forces engineered by the intensity gradient which ultimately can form a trap for small particles. Small dielectric particles for example feel an attractive force towards the point of maximum intensity.


The limitation of a point optical trap created using strongly focused laser beam with a high intensity gradient is its range. The distance over which the intensity gradient force can overcome the radiation force goes only up to the order of $\SI{1}{\mu m}$.This limitation revise the question if whether any mode of light  can manipulate objects over such a long ranges. The ideal optical mode for long range manipulation would be propagation invariant  and would create a net negative force field along its entire length when it interacts with an object. Previously, exotic modes were created which are spiraling around their axis were shown to act as a tractor beam \cite{Lee_2010}. But such tractor beams were realized over $\sim 100 \mathrm{\mu m}$ only. Here we present intermediate plane holography a new experimental technique for creating such tractor beams with increased power efficiency and larger propagation invariant range.


Due to the homology of paraxial wave equation with Schrodinger's equation the spatial structure of a propagating beam of light is analogous to the temporal evolution of a quantum mechanical wave packet. Therefore we can use optics to understand complex quantum mechanical states. For example a spiraling beam of light can created from a superposition of multiple Bessel beam with non-zero orbital angular momentum. An analogous quantum mechanical wave function can be described for a particle confined inside a infinite cylindrical well. In this thesis we have studied the propagation of a spiral beam to understand the evolution of  Bessel-like quantum state under a force-free condition, which constitutes an extraordinary class of self accelerating modes \cite{Berry1979}.

%First experimental realization of optical angular momentum was done by Beth \cite{beth1936}. 


\section{Organization}

This thesis is organized in the following manner:

Chapter \ref{ch:fundamentals_of_light} provides a brief historical overview of the properties of light followed by the basic formalism for describing electromagnetic waves. We introduce the generalized version of Maxwell's equations and arrive at the general solutions of the electric and the magnetic fields that solve the wave equations. After reviewing the mechanism by which such wave exert forces on small illuminated objects, we recast the problem of optical forces in terms of parameters used in our experiments and use this language to describe the propagation of light wave into the far field. 

Using the fundamental description of light, in Chapter \ref{ch:topological_modes}, then uses this formalism to  describe topological modes of light that can be generated using computational holography. The result of this analysis highlights a challenge. Topological modes of light have desirable properties for achieving long range transport, but they are essentially impossible to project with the conventional technologies of computational holography.

Chapter \ref{ch:intermediate} introduces a new technique called ``Intermediate Plane Holography (IPH)''  for structuring laser beams with computer-generated holograms. IPH can dramatically improve both diffraction efficiency and mode purity for challenging modes of light. We illustrate these capabilities by projecting Bessel beams, which constitute the natural basis for propagation-invariant modes \cite{durnin87,durnin87a}.

Chapter \ref{ch:accelerating} presents a new class of accelerating beam in two dimensions. Here we demonstrate that a shape-preserving wave packets that rotate at constant angular speed around the center of the box follows Ehrenfest's theorem. The apparent violation of Ehrenfest’s theorem is resolved by considering the force exerted on the particle’s wave packet by the enclosing wall.

Finally, Chapter 6 shows possible pathways for extending research in this field and its applications. (Yet to be done!!)
%\section{Outline of Dissertation}