\chapter{Conclusions}
\label{ch:conclusion}

This thesis has introduced a new technique, 
Intermediate-plane holography that is particularly useful for projecting
modes whose ideal Fresnel holograms are dominated by large
amplitude variations, and so suffer from low diffraction efficiency.
In addition to improving diffraction efficiency, shifting the
hologram plane also can improve mode purity by moving the
length scale for phase variations into the spatial bandwidth of a
practical diffractive optical element.
Both of these considerations figure in the success of intermediate-plane
holograms for projecting Bessel beams and their superpositions.
Because Bessel beams are the natural basis for propagation-invariant
modes, intermediate-plane holography lends itself naturally
to long-range projection.
We have demonstrated meter-scale projection using centimeter-scale
optical elements.
These same elements have additional potential applications for
topologically multiplexing and demultiplexing non-diffracting modes
for optical communications \cite{gibson04,bozinovic13,willner15}.
The same ability to project sophisticated superpositions of
topological modes could have additional applications to remote
sensing and LIDAR \cite{cvijetic15}.
Finally, the same principals discussed here in the context of
optical holography should apply equally well to other types
of waves, most notably to acoustic waves. As for future research,
intermediate plane holography can be utilized to create braided
or knotted optical field.

We believe that the material presented in this thesis will serve as a guide for further advancement in  designing optical modes of light using SLM with better mode purity and diffraction efficiency. It will also be a stepping stone to understand and realize kilometer scale tractor beam.