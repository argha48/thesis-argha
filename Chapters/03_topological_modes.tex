\chapter{Holographic Creation of Topological Modes of Light}
\label{ch:topological_modes}

\section{Topological Modes of Light}

\setlength{\epigraphwidth}{0.8\textwidth}
\epigraph{``Topology is the mathematical study of the properties that are preserved through deformations, twistings, and stretchings of objects \cite{topology}''}{\textit{Wolfram MathWorld, ``Topology''}}

It is the study of complex multidimensional curves and surfaces. By ``Topological Photonics'' \cite{leykam2016, Zhou_2017} majority of the researchers in the field of material science and hard condensed matter physics fundamentally think of discovering a new class of photonic-structure \cite{Barik666} that are able to transport light around sharp angles without back scattering. With these wave-guides scientists are able to realize exotic edge states with interesting properties that are found  in topological insulators \cite{hsieh2008, haldane20107}. While this field of research has applications in multiple areas \cite{ozawa2019} including photonic crystals, waveguides, metamaterials, cavities, optomechanics, silicon photonics, and circuit QED, we will digress from the generic meaning of the term. In this thesis we consider the study of global and local shape of the wavefront of a light field as ``Topological Photonics''. 

Many applications of structured light field require a mode of light with specified intensity distribution in a certain plane. The ability to focus a beam of light in a specific shape in space is useful for fields ranging from cryptography \cite{Horstmeyer2013, Horstmeyer2013cleo} to biology \cite{wang2012, Papadopoulos:12}, and neuroscience [XXXX ADD NEURO REFERENCE]. Therefore, understanding the topology of a wavefront is necessary as it determines the evolution of the electric field in space. 

%For example experimentally generated wavefront with low resolution will form local intensity structure upon propagation. On the other end of this spectrum, a wavefront beyond certain resolution can produce speckle field \cite{Indebetouw1993}.

Non-trivial wavefronts often feature phase singularities \cite{Bazhenov1992}.  One of the simplest examples of a topological defect in a mode of light is a screw dislocation \cite{Bazhenov1992} in the wavefront.
\begin{figure}[t!]
  \centering
  \includegraphics[width=0.7\textwidth]{helical_naturedgg}
  \caption{(a)A $\mathrm{TEM}_{00}$ mode can be converted into a helical wavefront by adding a $\ell \theta$ azimuthal phase ramp around its axis of propagation. (b) A optical vortex can be generated by focusing a helical wavefront with specified topological charge $\ell$. The radius of the optical vortex is directly related to the topological charge of the wavefront. (c) The orbital angular momentum carried by this mode of light can be transferred to illuminated particles either to make them rotate along the intensity maximum of the beam or to guide them along a designated trajectory as shown in  \cite{Curtis:03}. Reprinted with permission from \cite{grier2003nature}.}
  \label{fig:helical}
\end{figure}
This gives rise to a helical wavefront where the phase changes by $2m\pi$ upon one revolution around the axis of propagation. Here $m$ is an integer that describes the pitch of the wavefronts' curvature. Imposing a helical pitch on the phase introduces a helical mode's intensity distribution. Helical modes are dark \cite{Bazhenov1992,nye1997} on axis because all phase angles appear along the axis of the associated screw dislocation causing destructive interference. The beam's intensity is redistributed to a ring whose radius depends on the winding number $m$. Changing the topological charge therefore changes the intensity distribution of a helical mode of light. An ordinary $\mathrm{TEM_{00}}$ laser mode can be converted into a helical mode with spiral phase plate that imposes the required phase ramp in a particular plane. A spiral plate \cite{BEIJERSBERGEN1994321} also can convert a helical mode into another. Apart from redistributing the intensity distribution, helical modes' wavefront topology also redirects the light's momentum. The resulting spiral momentum converts into the beam with orbital angular momentum. A beam's orbital angular momentum depends only on its wavefront topology and is independent of its polarization. Linear polarized light that carries no spiral angular momentum still can carry orbital angular momentum. Apart from linear and spin angular momentum a helical wavefront carries rotational angular momentum. Such modes of light can be focused into optical traps to generate optical vortex. More sophisticated topological modes can even form knotted structures \cite{leach2004}. 


%A spiral phaseplate \cite{BEIJERSBERGEN1994321} can transform a  $\mathrm{TEM}_{00}$ mode of laser into a helical phasefront. They have topological phase singularities at the center of the wavefront \cite{Bazhenov1992,nye1997} which shows up as a dark spot in the intensity distribution at the center of the helical beam. All wavefronts in a helical mode of light superpose destructively to create such dark spot at the center along the axis of propagation. 



\section{Applications of Topological Modes}

Applications of topological modes of light was hindered by the difficulty of creating such modes using constrained optical elements such as mirrors and lenses. The introduction of diffractive optical elements effectively removed this barrieer and fostered a explosion of activity related to structured modes of light. The first widespread application of diffractive optics for beam shaping is the engraved into a plate of glass the Fresnel lens collects light from a large solid angle and collimates it into a beam without requiring a massive weight and size of a equivalent convex lens. The first such light shaping optics was installed in 1822. More recently, custom shaped laser beams have been used to address specific problems like in optical communication, photolithography, circuit component trimming, laser printing, optical data/image processing all of which requirement is that the light intensity is uniform over an area of cross-section \cite{Dickey03}. A collimated beam of light which is equivalent to a truncated plane wave is ideal for all such applications. 

One of the major challenges in the field of bio-medical optics is the limited range of imaging. Highly inhomogeneous distribution of refractive index inside a cell of most living organisms including human beings, reduces the spatial coherence of a light field. An added impurity to spatial coherence limits the ability to achieve diffraction limited focal spot size. Complex wavefront modulation to counteract such wavefront distortions due to propagation though turbid medium opens up new opportunities for optical micromanipulation in biological physics and also paves the way for super-resolution optical imaging. The idea of introducing adaptive optical elements to eliminate abberations is very well known in the field of astronomy \cite{Beuzit1997, beuzit1994}. Wavefront shaping techniques used in adaptive optics system, adjust and sharpen any blurriness formed in the image of a star due to atmospheric perturbation of the light field. Similar technique is followed in biomedical optics where the phase of the wavefront is corrected using spatial light modulator.

Optical vortices are generated by focusing a mode of light with helical phase-ramp which is different than a regular point optical tweezers created from Gaussian beam. A uniform optical vortex can exert torque on a trapped particle through orbital angular momentum transfer. This property can be leveraged to utilize optical vortex as particle sorter between absorbing and non-absorbing particles \cite{ONEIL2000139, Parkin:06, chavez2003}. Other desirable features of an optical vortex trap is its hollow structure and improved trapping efficiency in the axial direction \cite{NIEMINEN2008195}. Grier \emph{et al.} \cite{Curtis:03}  showed how multiple optical vortices can propel a polystyrene sphere along a specific trajectory in water. By changing the topological charge a dynamic optical vortex can be created with varying radius. 

The orthogonality of optical modes with different topological charges makes modes of light with helical wavefront highly applicable for multiplexing to increase data capacity of both free-space and fiber-optic communications \cite{Gibson:04, Willner_2016, Bozinovic1545, SHAO2018545}. Strong variation in the electric field near the phase singularity ``enables simultaneous single-spin imaging and magnetometry at the nanoscale with considerably less power than conventional techniques'' \cite{maurer2010}.

\section{Hermite-Gaussian and \\ Laguerre-Gaussian Modes}

The most common intended output of a laser cavity made by developers is a Gaussian beam. As its name implies Gaussian beam is an electromagnetic radiation whose transverse electric and magnetic amplitude profile is a Gaussian function. This Gaussian mode, which is also referred to as $\mathrm{TEM}_{00}$ mode, is one case of the generalized class of modes that are called ``Hermite-Gaussian (HG) modes'' which form a set of complete orthogonal basis functions that are also solutions of the paraxial Helmholtz equation in Cartesian coordinate system. The electric field of a HG mode, which is also denoted as $\mathrm{TEM}_{\ell m}$, can be written as:

\begin{equation}
\label{eq:HG beam}
\begin{split}
E_{\ell,m}(x,y,z) = & E_0 \frac{w_0}{w(z)} H_{\ell}\left( \frac{\sqrt{2}x}{w(z)}\right) H_{m}\left(\frac{\sqrt{2}y}{w(z)}\right) \times \\
& \exp \left(-\frac{x^2 + y^2}{w^2 (z)}\right) \exp \left(-i\frac{k(x^2 + y^2)}{2R(z)}\right) \exp (i\psi (z)) \quad ,
\end{split} \quad ,
\end{equation}
where $E_0$ is the normalization constant, $w_0$ is the diameter of the beam waist of the $\mathrm{TEM}_{00}$ mode, $w(z)$ and $R(z)$ are beam width and radius of curvature of the beam at an axial distance $z$ away from the beam waist, $\psi (z)$ is the Gouy phase. $H_{m}(\cdot)$ is the Hermite polynomial \cite{abramowitz+stegun} of order $m$. Because Hermite-Gaussian modes form the complete basis set of solutions to the paraxial Helmholtz equation, any arbitrary solution of the paraxial Helmholtz equation can be expressed as a superposition of multiple HG modes of light. A laser cavity with rectangular symmetry along the propagation axis can be adjusted to generate the family of HG beams. Many conventional strategies for generating topological modes of light, including helical modes atrt with Hermit-Gaussian modes and then impose special phase distributions with cyllindrical lenses or prisms.

\begin{figure}[t!]
  \centering
  \includegraphics[width=0.9\textwidth]{hglg}
  \caption{Hermite-Gaussian (HG) and Laguerre-Gaussian (LG) beams are solutions of the paraxial Helmholtz equation for Cartesian and cylindrical coordinates, respectively. The top row shows a few example of LG beams which have rotational symmetry, while the HG beams have rectangular symmetry as shown in the bottom row.}
  \label{fig:hglg}
\end{figure}


In cylindrical coordinates the same electric field in Eq.~\eqref{eq:HG beam} can be rewritten in terms of the generalized Laguerre polynomials. The associated basis functions are called Laguerre-Gaussian modes. Laser cavities with rotational symmetry produce such modes of light, where the electric field is written as:

\begin{equation}
\label{eq: LG Beam}
\begin{split}
E_{\ell p}(r,\phi, z) = & \frac{C^{LG}_{\ell p}}{w(z)}\left(\frac{r\sqrt{2}}{w(z)}\right) ^{|\ell|}\exp \left(-\frac{r^2}{w^2 (z)}\right) \times \\
							& L^{|\ell|}_{p}\left(\frac{2r^2}{w^2(z)}\right) \exp \left( -ik\frac{r^2}{2R(z)}\right) \times \\
							& \exp (-i\ell\phi) \exp (i\psi(z)) \quad ,
\end{split}
\end{equation}
where $L^{\ell}_{p} (\cdot)$ are the generalized Laguerre polynomials and $C^{LG}_{\ell p}$ is a normalization constant. $w(z)$, $R(z)$ and $\psi (z)$ have the same interpretation as in Eq.~\eqref{eq:HG beam}. Like HG beams, LG beams form the complete set of orthogonal basis functions that are solutions to the paraxial Helmholtz equation in cylindrical coordinates. A circularly symmetric mode of light, which is a solution to Helmholtz equation, can be decomposed into the superposition of multiple LG beams. The family of Laguerre-Gaussian beams has a helical phase profile parametrized by the topologicql charge $\ell$, and carries intrinsic orbital angular momentum of $\ell \hbar$ per photon \cite{allen1992} where $\ell$ is azimuthal mode index. In $\mathrm{1992}$ Allen \emph{et al.} drew attention to the possibility that LG modes could transfer angular momentum to illuminated objects thereby exerting torques as well as force. This prediction and its subsequent experimental demonstration spread interest in singular optics and inspired experimental initiatives in atom guiding and optical trapping. Optical traps based on Laguerre-Gaussian modes are also known as ``optical vortices'' 

Even though LG beams can be generated \emph{in situ} with a laser cavity with rotational symmetry, small aberrations can lead to loss of mode purity. This is why HG modes have been the preferred basis for generating LG modes using conventional optical elements. In 1994 M.W.Beijersbergen \emph{et al.} \cite{BEIJERSBERGEN1994321} showed experimentally how to convert a $\mathrm{TEM}_{00}$ mode into a helical wavefront using a spiral phase plate \cite{Ruffato:14} which takes the form of a circular ramp of glass. These refraction mode convertors are difficult to fabricate and can be replaced by equivalent diffractive elements that are reminiscent of Fresnel lenses with a twist. Not only are helical diffractive optical elements comparatively easy to fabricate in glass, they lend themselves to implementation with liquid-crystal display technologies, opening up the possibility of dynamic topological modes.


\section{Propagation Invariant Modes}

One of the limitations of both HG and LG beams is their tendency to diverge as they propagate. This diffraction renders them unsuitable for long-range optical micromanipulation. True long range optical micromanipulation requires the optical field to be propagation invariant \cite{TURUNEN20101} in the sense that the functional form of the transverse field distribution remains unchanged in free-space propagation \cite{bouchal1996}, although not necessarily with fixed orientation \cite{Piestun:98}. A plane wave is the simplest example of a propagation invariant field but cannot be projected in practical because it would carry infinite amount of energy. Imposing boundary condition in a plane wave to ontain a normalizable finite energy mode introduces diffraction and spoils its propagation invariance. In $1987$ Durnin and Micelli \cite{Durnin:87} first presented a class of solutions to paraxial Helmholtz  equation in cyllindrical coordinates, which are non-singular and propagation invariant. The simplest solution of this class of functions is:

\begin{equation}
\label{eq:Durnin J0}
\vec{E}(\vec{r}) = \exp (i\beta z)J_{0}(\alpha r) \hat{\epsilon} \quad ,
\end{equation}
where $J_0 (\cdot)$ is the zeroth order Bessel function of the first kind and $\alpha$ is a special parameter that determines the convergence angle of the plane waves that creates the Bessel field. One thing to notice in Eq.~\eqref{eq:Durnin J0} is that functional form of the electric field does not have a boundary condition. The transverse intensity goes down as $O(1/r)$ as $r\rightarrow \infty$ which manifests as a infinite energy carrying wavefront similar to a unbounded planar wave. Therefore, even though it is a solution of the paraxial Helmholtz equation in free space, the field described in Eq.~\eqref{eq:Durnin J0} is not realizable in real world. A finite version of Eq.~\eqref{eq:Durnin J0} will be truncated at certain $r = R_0$ which limits the range of propagation invariance.

\subsection{Bessel Beams}
\label{subsec:Bessel Beam}
The time-dependent electric field of a generalized Bessel beam \cite{bouchal1996, mcgloin2005} can be described as a complex field:

\begin{equation}
\label{eq:Jm_Bessel field}
\vec{E}(\vec{r},t) = J_{m}(k \sin \alpha \vec{r})e^{im\theta}e^{ik\cos \alpha z}e^{-i\omega t} \quad ,
\end{equation}
where $J_m (\cdot)$ is the $\mathrm{m^{th}}$ order Bessel function of the first kind, where $m$ is an integer. $k = \sqrt{k^2_z + k^2_r} = \frac{2\pi}{\lambda}$ is  the wavenumber of light, $r$, $\theta$ and $z$ are the radial, azimuthal and longitudinal coordinates respectively. $\alpha$ is a parameter that can take any value beteween $0$ and $\pi$ included. The amplitude of the Bessel beam has azimuthal symmetry and Eq.~\eqref{eq:Jm_Bessel field} satisfies the condition of propagation-invariant beam:

\begin{equation}
\label{eq:propagation invariant condition}
\frac{\partial}{\partial z}\vert \vec{E}(\vec{r},t)\vert ^2 = 0 \quad ,
\end{equation}

Apart from $m=0$ other Bessel beams carry orbital angular momentum which is unrelated to light's intrinsic linear and spin angular momentum. The amount of orbital angular momentum carried by individual photon is $m\hbar$, where $m$ is an integer that can take negative values as well. The Bessel beams are all cylindrically symmetric and, as the solutions of the paraxial Helmholtz equation, they form the complete basis set of orthogonal functions for an arbitrary cylindrically symmetric, propagation-invariant beam of light. Bessel beams naturally can be related to other sets of solutions to the Helmholtz equation, including those with symmetries. For example, they can be projected on to the set of plane waves through the identity:
 \begin{equation}
 \label{eq:Bessel Identity relation}
 J_m(r) = \frac{i^{-m}}{2\pi} \int _{0}^{2\pi}e^{im\theta}e^{ir\cos \theta }d\theta
 \end{equation}
The $m$th order Bessel beams thus can be a superposition of plane waves propagating at an angle that forms a cone centered around the optical axis. The vertex angle of the cone is related to the components of the wave vector through:
\begin{equation}
\label{eq:Bessel cone alpha}
\alpha = \tan ^{-1} \left(\frac{k_r}{k_z}\right)\quad ,
\end{equation}
where $k_r$ and $k_z$ are radial and longitudinal wave-vectors respectively. This angle, in turn controls the radius $r_0$ \cite{mcgloin2005} of the Bessel beam's central maxima through the relation:
\begin{equation}
\label{eq:Bessel core spot size}
r_0 = \frac{2.405}{k_z \tan \alpha}.
\end{equation}


\subsection{Other Propagation-Invariant Modes}

Since the discovery of Bessel beams in 1987, two other classes of propagation invariant modes of light has been reported. The first, proposed by Gutiérrez-Vega \emph{et al.} \cite{Gutierrez-Vega:00}, is called ``Mathieu beam''. These modes of light are described by the radial and angular Mathieu functions and are solution of the paraxial Helmholtz equation in the elliptical cylindrical coordinate system. The Mathieu beam of light propagates along elliptical trajectories. Another class of propagation-invariant mode is called the ``Weber beam'' \cite{Bandres_2013}. Unlike Mathieu beams, Weber waves propagate in parabolic trajectories. Airy beams are a special case of such modes of light.

Both Mathieu and Weber modes are described as accelerating modes of light because their intensity maxima trace out nonlinear paths. This remarkable property has excited considerable interest both for potential practical applications and also because such force free acceleration appears to violate conservation of momentum for light and Ehrenfest's theorem by analogy to quantum wave particles. These apparent anomalies are discusses and resolved in Chapter.~\eqref{ch:accelerating}.

\section{Optical Tweezers}
\begin{figure}[t!]
  \centering
  \includegraphics[width=0.6\textwidth]{grier2013nature}
  \caption{A beam of light, tightly focused using a high numerical aperture creates a strong intensity gradient. It acts as an attractive force and drags small colloidal particles towards the focus spot. Whereas the radiation pressure repels the particle and push it away from the focal spot. When gradient force supersedes radiation pressure a stable optical trap is created and a particle can be trapped in three dimension near the focal spot. Picture taken with permission from David G. Grier \cite{grier2003nature}}
  \label{fig:Optical tweezers}
\end{figure}

Tweezers are beams of light that can trap microscopic objects in three dimensions. They are usually created by focusing a laser beam tightly, to the diffraction limit, thereby maximuzung the ability of intensity gradient forces to effectively localize a dielectric particle despite repulsive radiation pressure and other external forces. In 1986 Arthur Ashkin and his co-workers \cite{Ashkin:86} from Bell Laboratories reported their experimental discovery that a single focused beam can trap particles. Their explanation of this surprising observation launched the field of optical trapping. Due to this pioneering experiment and his contribution in the field of optical tweezers Arthur Ashkin was awarded the 2018 Nobel Prize in Physics \cite{nobel_media_2019}. Since the inception, many physicists and biologists have made excellent use of this versatile technique \cite{grier2003nature}. The optical force on a trapped object can be controlled at an \SI{100}{\mathrm{a}\newton} precision between roughly \SI{1}{\femto \newton} to as much as \SI{100}{\mathrm{p}\newton} \cite{Rohrbach:02}. Such range of forces is ideal for probing biological systems, ranging from single molecule biophysics to measuring responses in macromolecular systems \cite{Svoboda1994,Litvinov7426,Brouhard2003}. Svoboda \emph{et al.}  \cite{Svoboda11782} has used optical tweezers to study single molecules of the motor protein kinesin, moving under low mechanical loads at saturating ATP concentrations. Mondal \emph{et al.} \cite{argha2014} has discovered how weakly focused beam can be used for highly efficient axonal guidance in a non-invasive manner. Outside of biological applications, optical tweezers have been used in myriad of studies including measurements of pair interaction potential of charge-stabilized colloid \cite{crocker1994}, for characterizing and tracking single colloidal particles \cite{Lee07colloid, Cheong2009, xiao2010, chen2015, chen2016} and for probing nonequilibrium statistical physics [ADD REFERENCE]


\section{Holographic Optical Trapping}
\label{sec:HOT}

As useful as optical tweezers have been, they have the limitation that one beam of light creates one trap and that trap can only act as a static potential energy well. Holographic Optical Traps (HOT) modifies the wavefront of a single light beam using a computer generated hologram to create different mode structures that focus into more general trapping patterns, including three dimensional arrangement of optical tweezers, optical vortices, Bessel beams and more. If you look back at Eq.~\eqref{eq:E_amp_phase} the optical field can be controlled either by changing the amplitude $u(\vec{r})$ or by modifying the phase $\phi (\vec{r})$. In our present setup we utilize a phase-only Spatial Light Modulator (SLM) \cite{Igasaki1999} to adjust the phase of the field in a plane as described in \cite{he1995, dufrense2001hot, CURTIS2002169, Grier:06, Polin:05}. A schematic of our experimental setup is shown in Fig.~\ref{fig:HOTsetup}

\begin{figure}[t!]
  \centering
  \includegraphics[width=0.7\textwidth]{setup}
  \caption{Schematic of HOT setup. Permission: copied from Bhaskar's thesis, [MAKE YOUR OWN FIG]}
  \label{fig:HOTsetup}
\end{figure}

We use a linearly polarized laser at a wavelength of \SI{532}{\nm} (Coherent Verdi $\mathrm{5W}$) as our trapping laser. The initial diameter of the laser beam is less than \SI{2}{\mm}. It passes through a 5\texttimes beam expander before it incidents on the phase-only SLM (Hamamatsu $\mathrm{X10468-16}$). The 5\texttimes beam expander makes the beam uniform and overfills the aperture (\SI{15.8}{\mm}\texttimes \SI{12}{\mm}) of the SLM. The incident angle is maintained at less than $8\degree$ in order to retain the maximum light utilization efficiency ($96\%$) of the reflective SLM. The SLM has a $1920\times1080$ pixel liquid crystal screen which limits the resolution of the computer generated holograms that can be imprinted on the light field. The resultant light from the SLM then goes through a telescopic ($4-f$) system before going through the objective lens. Depending on trapping requirements, we use a high numerical aperture (NA) oil immersion objective (Nikon Plan-Apo 100\texttimes; NA 1.4). Finally the light is focused inside the sample chamber, which for our purpose consist of aqueous colloid dispersion contained in glass microfluidic chambers.


\section{Phase-Only Holograms}
\label{sec:PhaseHOT}
\begin{figure}[t!]
  \centering
  \includegraphics[width=0.75\textwidth]{fourier}
  \caption{To create a specific mode of light with electric field $\vec{E}^{f}(\vec{\rho})$ on the sample plane using digital holography, we need to impose a phase pattern in the input hologram plane that will transform the laser field into $\vec{E}^{in}(\vec{r})$. Due to the specific configuration of the setup shown in this schematic $\vec{E}^{f}(\vec{\rho}) = \mathcal{F}\lbrace E^{\mathrm{in}}(\vec{r}) \rbrace$.}
  \label{fig:fourier}
\end{figure}

The experimental setup shown in Fig.~\ref{fig:HOTsetup} can be simplifed into Fig.~\ref{fig:fourier} in order to understand the relation between the optical field in the SLM plane and the field in the sample plane. In the paraxial limit, the lens acts as a Fourier transform operator \cite{goodmanfourier}. The electric field in the focal plane ($E^{f}(\vec{\rho})$) of the objective lens is related to the field in the hologram plane ($E^{\mathrm{in}}(\vec{r})$) by:
\begin{subequations}
\label{eq:fourier_relation}
\begin{align}
        E^{f}(\vec{\rho}) & = -\frac{i}{\lambda f} \int dr^{2} E^{\mathrm{in}}(\vec{r}) \exp \left(-i\frac{k}{f}\vec{r}\cdot \vec{\rho}\right) \quad , \\
 & \equiv \mathcal{F} \lbrace E^{\mathrm{in}}(\vec{r}) \rbrace \quad \mathrm{and}
\end{align}
\end{subequations}

\begin{equation}
\label{eq:inversefourier}
E^{\mathrm{in}}(\vec{r})   = \mathcal{F}^{-1}\lbrace E^{f}(\vec{\rho})\rbrace
\end{equation}
where $f$ is the focal length of the objective lens and $\vec{r}$ and $\vec{\rho}$ are the position vectors on the hologram plane and on the focal plane respectively. Therefore, in order to obtain a desired electric field in the focal plane we need the inverse Fourier transformed electric field in the hologram plane. Now to get $E^{\mathrm{in}}(\vec{r})$ from the incident laser field, $E_{0}(\vec{r})\equiv u_{0}(\vec{r})e^{i\phi _{0}(\vec{r})}$ we need to modify both the amplitude and the phase of the input field in the hologram plane. While several research groups have devised ways to modify the amplitude profile \cite{Ando:09, Bagnoud:04, Liu:14, Wilson:07, vanPutten:08, zhu2014} using phase-only SLM, for our purpose such techniques do not improve the mode quality or diffraction efficiency by enough to justify the additional expense and complexity. 

Let us assume, that to obtain the desired electric field $F^{r}(\vec{\rho})$ we need to transform the incident laser field $E_{0}(\vec{r})$ into $E^{\mathrm{in}}(\vec{r}) = u^{\mathrm{in}}(\vec{r})\exp (ik\phi ^{\mathrm{in}}(\vec{r}))$ in the hologram plane. We overfill the SLM aperture in order to achieve real uniform illumination $u_{0}(\vec{r}) = 1$. Using our phase-only SLM, we can transform the phase of the incident field $\phi _{0}(\vec{r})$ into $\phi _{\mathrm{in}}(\vec{r})$. Therefore we end up with a residual electric field equal to the second term of Eq.~\eqref{eq:extrafield}
\begin{subequations}
\label{eq:extrafield}
\begin{align}
e^{\phi _{\mathrm{in}}(\vec{r})} \quad = \quad & u^{\mathrm{in}}(\vec{r}) e^{ik\phi ^{\mathrm{in}}(\vec{r})} \\
											&  + \left( 1- u^{\mathrm{in}}(\vec{r})\right) e^{ik\phi ^{\mathrm{in}}(\vec{r})}
\end{align}
\end{subequations}
The first term on the right side of Eq.~\eqref{eq:extrafield} is the desired firld in the hologram plane. The second term represents additional light that is not in the desired mode. To prevent any interference between these two terms we deflect the desired light field by adding an additional phase ramp \cite{CURTIS2002169} equal to :
\begin{equation}
\phi _{\mathrm{disp}}(\vec{r}) = \frac{k}{f}\vec{\rho _{0}}\cdot \vec{r}
\end{equation}
to position $\vec{\rho _{0}}$ in the trapping plane. The offset $\rho _{0}$ can be chosen to minimize any overlap between the intended trapping pattern and artifacts due to errors in the amplitude profile. More sophisticated algorithms can correct for amplitude errors by modifying the projected phase profile, trading off phase errors for diffraction efficiency, amplitude uniformity or other figures of merit. For the purpose of this thesis, simple displacement is sufficient and conserves high fidelity of the projected mode at the cost of minimal optimization of the displacement.




