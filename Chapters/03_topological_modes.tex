\chapter{Holographic Creation of Topological Modes of Light}
\label{ch:topological_modes}

\section{Topological Modes of Light}

\setlength{\epigraphwidth}{0.8\textwidth}
\epigraph{``Topology is the mathematical study of the properties that are preserved through deformations, twistings, and stretchings of objects \cite{topology}''}{\textit{Wolfram MathWorld, ``Topology''}}

It is the study of complex multidimensional curves and surfaces. By ``Topological Photonics'' \cite{leykam2016, Zhou_2017} majority of the researchers in the field of material science and hard condensed matter physics fundamentally think of discovering a new class of photonic-structure \cite{Barik666} that are able to transport light around sharp angles without back scattering. With these wave-guides scientists are able to realize exotic edge states with interesting properties that are found  in topological insulators \cite{hsieh2008, haldane20107}. While this field of research has applications in multiple areas \cite{ozawa2019} including photonic crystals, waveguides, metamaterials, cavities, optomechanics, silicon photonics, and circuit QED, we will digress from the generic meaning of the term. In this thesis we consider the study of global and local shape of the wavefront of a light field as ``Topological Photonics''. 

Many applications of structured light field require a mode of light with specified intensity distribution in a certain plane. The ability to focus a beam of light in a specific shape in space has a huge potential for applications in fields ranging from cryptography \cite{Horstmeyer2013, Horstmeyer2013cleo} to biology \cite{wang2012, Papadopoulos:12}. Therefore, understanding the topology of a wavefront is necessary as it determines the evolution of the electric field in space. 

A non-trivial wavefront is prone to contain a phase defect \cite{Bazhenov1992}. For example experimentally generated wavefront with low resolution will form local intensity structure upon propagation. On the other end of this spectrum, a wavefront beyond certain resolution can produce speckle field \cite{Indebetouw1993}. One of the simplest example of topological defect in a mode of light is a screw dislocation \cite{Bazhenov1992} in the wavefront.
\begin{figure}[t!]
  \centering
  \includegraphics[width=0.7\textwidth]{helical_naturedgg}
  \caption{(a)A $\mathrm{TEM}_{00}$ mode can be converted into a helical wavefront by adding a $m\theta$ azimuthal phase ramp around its axis of propagation. (b) Dynamic optical vortex can be generated by focusing a helical wavefront with varying topological charge $m$. The radius of the optical vortex is directly related to the topological charge of the wavefront. (c)The orbital angular momentum carried by this mode of light can be transferred into particles to either make them rotate along the intensity maximum of the beam or guide them along certain trajectory as shown in  \cite{Curtis:03}. Reprinted with permission from \cite{grier2003nature}.}
  \label{fig:helical}
\end{figure}
This gives rise to a helical wavefront where the phase changes by $2m\pi$ upon one revolution around the axis of dislocation while propagating in free space. Here $m$ is an integer, which is called the topological charge. A change in the topological charge updates the intensity distribution of the helical modes of light. A spiral phaseplate \cite{BEIJERSBERGEN1994321} can transform a  $\mathrm{TEM}_{00}$ mode of laser into a helical phasefront. They have topological phase singularities at the center of the wavefront \cite{Bazhenov1992,nye1997} which shows up as a dark spot in the intensity distribution at the center of the helical beam. All wavefronts in a helical mode of light superpose destructively to create such dark spot at the center along the axis of propagation. Apart from linear and spin angular momentum a helical wavefront carries rotational angular momentum. Such modes of light can be focused into optical traps to generate optical vortex. Optical vortices can even form knotted structures \cite{leach2004}. 


\section{Applications of Topological Modes}

The first useful wavefront modulation was done using beam shaping tool Fresnel lens used in light houses. It allowed people to collect light from a large solid angle without having to create a ginormous convex lens and focus it at a long distance so that it is visible from deep in the ocean. More recently, custom shaped laser beam have been used to address specific problems like in optical communication, photo-lithography, circuit component trimming, laser printing, optical data/image processing where a similar requirement is that the light intensity is uniform over an area of cross-section \cite{Dickey03}. A collimated beam of light which is equivalent to truncated plane waves is ideal for such applications. 

One of the major challenges in the field of bio-medical optics is the limited range of imaging. Highly inhomogeneous distribution of refractive index inside a cell of most living organisms including human beings, reduces the spatial coherence of a light field. An added impurity to spatial coherence limits the ability to achieve diffraction limited focal spot size. Complex wavefront modulation to counteract such wavefront distortions due to propagation though turbid medium opens up new opportunities for optical micromanipulation in biological physics and also paves the way for super-resolution optical imaging. The idea of introducing adaptive optical elements to eliminate abberations is very well known in the field of astronomy \cite{Beuzit1997, beuzit1994}. Wavefront shaping techniques used in adaptive optics system, adjust and sharpen any blurriness formed in the image of a star due to atmospheric perturbation of the light field. Similar technique is followed in biomedical optics where the phase of the wavefront is corrected using spatial light modulator.

Optical vortices are generated by focusing a mode of light with helical phase-ramp which is different than a regular point optical tweezers created from Gaussian beam. A uniform optical vortex can exert torque on a trapped particle through orbital angular momentum transfer. This property can be leveraged to utilize optical vortex as particle sorter between absorbing and non-absorbing particles \cite{ONEIL2000139, Parkin:06, chavez2003}. Other desirable features of an optical vortex trap is its hollow structure and improved trapping efficiency in the axial direction \cite{NIEMINEN2008195}. Grier \emph{et al.} \cite{Curtis:03}  showed how multiple optical vortices can propel a polystyrene sphere along a specific trajectory in water. By changing the topological charge a dynamic optical vortex can be created with varying radius. 

The orthogonality of optical modes with different topological charges makes modes of light with helical wavefront highly applicable for multiplexing to increase data capacity of both free-space and fiber-optic communications \cite{Gibson:04, Willner_2016, Bozinovic1545, SHAO2018545}. Strong variation in the electric field near the phase singularity ``enables simultaneous single-spin imaging and magnetometry at the nanoscale with considerably less power than conventional techniques'' \cite{maurer2010}.

\section{Hermite-Gaussian and \\ Laguerre-Gaussian Modes}

The most common intended output of a laser cavity made by developers is a Gaussian beam. As its name implies Gaussian beam is an electromagnetic radiation whose transverse electric and magnetic amplitude profile is a Gaussian function. This Gaussian mode, which is also referred to as $\mathrm{TEM}_{00}$ mode, is one case of the generalized class of modes that are called ``Hermite-Gaussian (HG) modes'' which form a set of complete orthogonal basis functions that are also solutions of the paraxial Helmholtz equation in Cartesian coordinate system. The electric field of a HG mode, which is also denoted as $\mathrm{TEM}_{lm}$, can be written as:

\begin{equation}
\label{eq:HG beam}
\begin{split}
E_{l,m}(x,y,z) = & E_0 \frac{w_0}{w(z)} H_{l}\left( \frac{\sqrt{2}x}{w(z)}\right) H_{m}\left(\frac{\sqrt{2}y}{w(z)}\right) \times \\
& \exp \left(-\frac{x^2 + y^2}{w^2 (z)}\right) \exp \left(-i\frac{k(x^2 + y^2)}{2R(z)}\right) \exp (i\psi (z)) \quad ,
\end{split}
\end{equation}
$E_0$ is the normalization constant, $w_0$ is the beam waist of the $\mathrm{TEM}_{00}$ mode, $w(z)$ and $R(z)$ are beam width and radius of curvature of the beam at an axial distance $z$ away from the beam waist. The second and the third term are Hermite polynomials \cite{abramowitz+stegun} of order $l$ and $m$ respectively. $\psi (z)$ is the Gouy phase at $z=z$. Because Hermite-Gaussian modes form the complete basis set of solutions for the paraxial Helmholtz equation, any arbitrary solution of the paraxial Helmholtz equation can be expressed as a superposition of multiple HG modes of light. Stable laser cavity with rectangular symmetry along the propagation axis are capable of generating the family of HG beams.

\begin{figure}[t!]
  \centering
  \includegraphics[width=0.9\textwidth]{hglg}
  \caption{Hermite-Gaussian (HG) and Laguerre-Gaussian (LG) beams are solutions of the paraxial Helmholtz equation for Cartesian and cylindrical coordinates, respectively. The top row shows a few example of LG beams which have rotational symmetry, while the HG beams have rectangular symmetry as shown in the bottom row.}
  \label{fig:hglg}
\end{figure}


In cylindrical coordinate system the same electric field in Eq.~\eqref{eq:HG beam} can be rewritten in terms of the generalized Laguerre polynomials, which are called Laguerre-Gaussian beam. Laser cavity with rotational symmetry produces such modes of light, where the electric field is written as:

\begin{equation}
\label{eq: LG Beam}
\begin{split}
E_{lp}(r,\phi, z) = & \frac{C^{LG}_{lp}}{w(z)}\left(\frac{r\sqrt{2}}{w(z)}\right) ^{|l|}\exp \left(-\frac{r^2}{w^2 (z)}\right) \times \\
							& L^{|l|}_{p}\left(\frac{2r^2}{w^2(z)}\right) \exp \left( -ik\frac{r^2}{2R(z)}\right) \times \\
							& \exp (-il\phi) \exp (i\psi(z)) \quad ,
\end{split}
\end{equation}

where $L^{l}_{p}$ are the generalized Laguerre polynomials and $C^{LG}_{lp}$ is the normalization constant. $w(z)$, $R(z)$ has the same definition as in Eq.~\eqref{eq:HG beam}. Alike HG beams, LG beams form the complete set of orthogonal basis functions that are solution to the paraxial Helmholtz equation in cylindrical coordinate. A circularly symmetric mode of light, which is a solution to Helmholtz equation, can be decomposed into the superposition of multiple LG beams. These family of Laguerre-Gaussian beams carry intrinsic orbital angular momentum of $l\hbar$ per photon \cite{allen1992} where $l$ is azimuthal mode index. In $\mathrm{1992}$ Allen \emph{et al.} outlined the amount of angular momentum transfer possible to the interacting object and converted into mechanical torque. Due to this nature the LG beams are of considerable practical interest in atom guiding and optical trapping. Laguerre-Gaussian optical tweezers are also known as ``optical vortices'' 

Even though the LG beams can be generated using laser cavity with rotational symmetry along the axis of propagation, a small aberration can lead to loss of mode purity. This is why HG modes are dominant mode of output used while building a laser. In 1994 M.W.Beijersbergen \emph{et al.} \cite{BEIJERSBERGEN1994321} showed experimentally how to convert a $\mathrm{TEM}_{00}$ mode into a helical wavefront using a spiral phaseplate \cite{Ruffato:14}.


\section{Propagation Invariant Modes}

One of the limitations of both HG and LG beams is their diverging property. Gaussian beams are not suitable for long-range optical micro-manipulation. Such long-range optical micro-manipulation requires the optical field to be propagation invariant \cite{TURUNEN20101}. A necessary condition of the wave field is the functional form of the transverse field distribution and the scale of intensity distribution remains unchanged in free-space propagation \cite{bouchal1996}, although not necessarily with the same orientation \cite{Piestun:98}. By definition it is not a requirement for such modes of light to be continuously propagation invariant. A plane wave is the simplest example of a propagation invariant field which carries infinite amount of energy in the absence of a boundary. It is impossible to create such a field and added boundary condition becomes evident in the form of diffraction pattern as the plane wave propagated in free space. In $1987$ Durnin \cite{Durnin:87} first presented a class of solution of paraxial Helmholtz  equation, which are non-singular and the functional form of the transverse field in a plane is unaltered by propagating in free space. The simplest solution of this class of functions is:

\begin{equation}
\label{eq:Durnin J0}
\vec{E}(\vec{r}) = \exp (i\beta z)J_{0}(\alpha \vec{r}) \quad ,
\end{equation}

where $J_0$ is the zeroth order Bessel function of the first kind and $\alpha$ is a special parameter that determines the convergence angle of the plane waves that creates the Bessel field. One thing to notice in Eq.~\eqref{eq:Durnin J0} is the functional form of the electric field does not have a boundary condition. The transverse intensity goes down as $O(1/r)$ as $r\rightarrow \infty$ which manifests as a infinite energy carrying wavefront similar to a unbounded planar wave. Therefore, even though it is a solution of the paraxial Helmholtz equation in free space, the field described in Eq.~\eqref{eq:Durnin J0} is not realizable in real world. A finite version of Eq.~\eqref{eq:Durnin J0} will be truncated at certain $r = R_0$ which limits the range of propagation invariance.

\subsection{Bessel Beam}
\label{subsec:Bessel Beam}
The time-dependent electric field of a generalized Bessel beam \cite{bouchal1996, mcgloin2005} can be described as a complex field:

\begin{equation}
\label{eq:Jm_Bessel field}
\vec{E}(\vec{r},t) = J_{m}(k \sin \alpha \vec{r})e^{im\theta}e^{ik\cos \alpha z}e^{-i\omega t} \quad ,
\end{equation}

where $J_m$ is the $\mathrm{m^{th}}$ order Bessel function of the first kind, where $m$ is an integer. $k = \sqrt{k^2_z + k^2_r} = \frac{2\pi}{\lambda}$ is  wavenumber, $r$, $\theta$ and $z$ are the radial, azimuthal and longitudinal components respectively. $\alpha$ is a parameter that can take any value beteween $0$ and $\pi$ included. The amplitude of the Bessel beam has azimuthal symmetry and Eq.~\eqref{eq:Jm_Bessel field} satisfy the condition of propagation-invariant beam:

\begin{equation}
\label{eq:propagation invariant condition}
I(r,z\geqslant 0) = I(r) \quad ,
\end{equation}
where $I$ is the transverse intensity of the beam.

 Apart from $m=0$ other Bessel beams carry orbital angular momentum which is unrelated to light's intrinsic linear and spin angular momentum. The amount of orbital angular momentum carried by individual photon is $m\hbar$, where $m$ is an integer that can take negative values as well. The Bessel beams are all cylindrically symmetric and they form the complete basis set of orthogonal functions for an arbitrary cylindrically symmetric, propagation-invariant beam of light. The identity relation of Bessel function:
 
 \begin{equation}
 \label{eq:Bessel Identity relation}
 J_m(r) = \frac{i^{-m}}{2\pi} \int _{0}^{2\pi}e^{im\theta}e^{ir\cos \theta }d\theta
 \end{equation}
garners our attention to an alternative interpretation of the Bessel beam of light. It can be decomposed into plane waves propagating at an angle that forms a cone shape. The vertex angle of the cone:
\begin{equation}
\label{eq:Bessel cone alpha}
\alpha = \tan ^{-1}\frac{k_r}{k_z}\quad ,
\end{equation}
where $k_r$ and $k_z$ are radial and longitudinal wave-vectors respectively, determines the size of the center spot. The radius $r_0$ \cite{mcgloin2005} of the center spot is related to $\alpha$ through the relation:
\begin{equation}
\label{eq:Bessel core spot size}
r_0 = \frac{2.405}{k_z \tan \alpha}.
\end{equation}


\subsection{Other Propagation-Invariant Modes}

There are two other classes of beams that fall into the same category of propagation-invariant modes of light. The first one was first proposed by Gutiérrez-Vega \emph{et al.} \cite{Gutierrez-Vega:00} which is called ``Mathieu beam''. These modes of light are described by the radial and angular Mathieu functions. These are solution of the paraxial Helmholtz equation in the elliptical cylindrical coordinate system. The Mathieu beam of light propagates along elliptical trajectories. Another class of propagation-invariant mode is called ``Weber beam'' \cite{Bandres_2013}. Unlike Mathieu beams, Weber waves propagate in parabolic trajectories. Airy beam is a special case of such modes of light.

These modes of light are also called accelerating modes of light due to their nonlinear path of propagation, which we talk in details in Ch.~\eqref{ch:accelerating}. While researchers have already experimentally created these mode of light, they are beyond the scope of this thesis because of their non-linear nature of propagation and difficult to create.

\section{Optical Tweezers}
\begin{figure}[t!]
  \centering
  \includegraphics[width=0.6\textwidth]{grier2013nature}
  \caption{A beam of light, tightly focused using a high numerical aperture creates a strong intensity gradient. It acts as an attractive force and drags small colloidal particles towards the focus spot. Whereas the radiation pressure repels the particle and push it away from the focal spot. When gradient force supersedes radiation pressure a stable optical trap is created and a particle can be trapped in three dimension near the focal spot. Picture taken with permission from David G. Grier \cite{grier2003nature}}
  \label{fig:Optical tweezers}
\end{figure}

Optical tweezers can impart customizable forces on the trapped colloidal particles. The ability to focus a laser beam tightly, to the diffraction limit, offers us the control to effectively trap a dielectric particle at length scales ranging from few nanometers to millimeters. In 1986 Arthur Ashkin and his co-workers \cite{Ashkin:86} from Bell laboratories for the first time reported how a single beam can be tightly focused to create a strong gradient force to trap particles. Due to this pioneering experiment and his contribution in the field of optical tweezers Arthur Ashkin won the 2018 Nobel Prize in Physics\cite{nobel_media_2019}. Since the inception, many physicist and biologist have shown the versatility of this technique\cite{grier2003nature}. The optical force on a trapped object can be controlled at an $\sim 100 \mathrm{aN}$ precision between $\sim 1 \mathrm{pN}$ to a few $\sim 100\mathrm{pN}$ \cite{Rohrbach:02}. Such range of force is optimal for probing biological systems, ranging from single molecule biophysics to measuring responses in macromolecular systems\cite{Svoboda1994,Litvinov7426,Brouhard2003}. Svoboda \emph{et al.} \cite{Svoboda11782} has studied single molecules of the motor protein kinesin, moving under low mechanical loads at saturating ATP concentrations. Mondal \emph{et al.} \cite{argha2014} has discovered how weakly focused beam can be used for highly efficient axonal guidance in a non-invasive manner. Outside of biological applications, optical tweezers have been used in myriad of studies for example, to measure pair interaction potential of charge-stabilized colloid \cite{crocker1994}, for characterizing and tracking single colloidal particles \cite{Lee07colloid, Cheong2009, xiao2010, chen2015, chen2016}.


\section{Holographic Optical Trapping}
\label{sec:HOT}
In Holographic Optical tweezers the wavefront of a single light beam is modified using a computer generated hologram to create different mode structures. If you look back at Eq.~\eqref{eq:E_amp_phase} the optical field can be controlled either by changing the amplitude $u(\vec{r})$ or by modifying the phase $\phi (\vec{r})$. In our present setup we utilize a phase-only Spatial Light Modulator (SLM) \cite{Igasaki1999} to adjust the phase of the field in a plane as it is described in \cite{he1995, dufrense2001hot, CURTIS2002169, Grier:06, Polin:05}. A schematic of our experimental setup is shown in Fig.~\ref{fig:HOTsetup}

\begin{figure}[t!]
  \centering
  \includegraphics[width=0.7\textwidth]{setup}
  \caption{Schematic of HOT setup. Permission: copied from Bhaskar's thesis, will make my own}
  \label{fig:HOTsetup}
\end{figure}

We use a linearly polarized laser at a wavelength of $532 \mathrm{nm}$ from Coherent Verdi $\mathrm{5W}$ as our trapping laser. The initial diameter of the laser beam is less than $2\mathrm{mm}$. It passes through a $\mathrm{5X}$ beam expander before it incidents on the phase-only SLM (Hamamatsu $\mathrm{X10468-16}$). The $5\mathrm{X}$ beam expander makes the beam uniform and overfills the aperture ($15.8\mathrm{mm}\times 12\mathrm{mm}$) of the SLM. The incident angle is maintained at less than $5\degree$ in order to retain the maximum light utilization efficiency ($96\%$) of the reflective SLM. The SLM has a $1920\times1080$ pixels liquid crystal screen which limits the resolution of the computer generated holograms that can be imprinted on the light field. The resultant light from the SLM then goes through a telescopic ($4-f$) system before going through the objective lens. Depending on trapping requirements, we use a high numerical aperture (NA) oil immersion objective (Nikon Plan-Apo 100X; NA 1.4). Finally the light is focused inside the sample chamber, which for our purpose consist of water solution of colloidal particle of $\sim 2\mathrm{\mu m}$ in diameter.


\section{Phase-Only Holograms}
\label{sec:PhaseHOT}
\begin{figure}[t!]
  \centering
  \includegraphics[width=0.75\textwidth]{fourier}
  \caption{In order to create a specific mode of light with electric field $\vec{E}^{f}(\vec{\rho})$ on the sample plane using digital holography, we need to project a phase in the input hologram plane that will transform the Laser field into $\vec{E}^{in}(\vec{r})$. Due to the specific configuration of the setup shown in this schematic $\vec{E}^{f}(\vec{\rho}) = \mathcal{F}\lbrace E^{\mathrm{in}}(\vec{r}) \rbrace$.}
  \label{fig:fourier}
\end{figure}

The experimental setup shown in Fig.~\ref{fig:HOTsetup} can be simplifed into Fig.~\ref{fig:fourier} in order to understand the relation between the optical field in the SLM plane and the field in the sample plane. In the paraxial limit the lens acts as a Fourier transform operator \cite{goodmanfourier}. The electric field in the focal plane ($E^{f}(\vec{\rho})$) of the objective lens is related to the field in the hologram plane ($E^{\mathrm{in}}(\vec{r})$) by:

%\begin{equation}
%\label{eq:fourier_relation}
%\begin{split}
%E^{f}(\vec{\rho}) & = -\frac{i}{\lambda f} \int dr^{2} E^{\mathrm{in}}(\vec{r}) \exp \left(-i\frac{k}{f}\vec{r}\cdot \vec{\rho}\right) \quad , \\
% & \equiv \mathcal{F} \lbrace E^{\mathrm{in}}(\vec{r}) \rbrace
%\end{split}
%\end{equation}

\begin{subequations}
\label{eq:fourier_relation}
\begin{align}
        E^{f}(\vec{\rho}) & = -\frac{i}{\lambda f} \int dr^{2} E^{\mathrm{in}}(\vec{r}) \exp \left(-i\frac{k}{f}\vec{r}\cdot \vec{\rho}\right) \quad , \\
 & \equiv \mathcal{F} \lbrace E^{\mathrm{in}}(\vec{r}) \rbrace \quad \mathrm{and}
\end{align}
\end{subequations}

\begin{equation}
\label{eq:inversefourier}
E^{\mathrm{in}}(\vec{r})   = \mathcal{F}^{-1}\lbrace E^{f}(\vec{\rho})\rbrace
\end{equation}

where $f$ is the focal length of the objective lens and $\vec{r}$ and $\vec{\rho}$ are the position vectors on the hologram plane and on the focal plane respectively. Therefore, in order to obtain a desired electric field in the focal plane we need the inverse Fourier transformed electric field in the hologram plane. Now to get $E^{\mathrm{in}}(\vec{r})$ from the incident laser field $E_{0}(\vec{r})\equiv u_{0}(\vec{r})e^{i\phi _{0}(\vec{r})}$ we need to modify both amplitude and the phase of the input field in the hologram plane. While several research groups have devised ways to modify the amplitude \cite{Ando:09, Bagnoud:04, Liu:14, Wilson:07, vanPutten:08, zhu2014} using phase-only SLM, for our purpose such techniques do not improve the mode quality by reasonable amount. Let us assume, in order to obtain the desired electric field $F^{r}(\vec{\rho})$ we need to transform incident laser field $E_{0}(\vec{r})$ into $E^{\mathrm{in}}(\vec{r}) = u^{\mathrm{in}}(\vec{r})\exp (ik\phi ^{\mathrm{in}}(\vec{r}))$ in the hologram plane. We overfill the SLM aperture in order to make the assumption $u_{0}(\vec{r}) = 1$. Using our phase-only SLM we can transform the phase of the incident field $\phi _{0}(\vec{r})$ into $\phi _{\mathrm{in}}(\vec{r})$. Therefore we end up with a residual electric field equal to the second term of Eq.\eqref{eq:extrafield}

\begin{subequations}
\label{eq:extrafield}
\begin{align}
e^{\phi _{\mathrm{in}}(\vec{r})} \quad = \quad & u^{\mathrm{in}}(\vec{r}) e^{ik\phi ^{\mathrm{in}}(\vec{r})} \\
											&  + \left( 1- u^{\mathrm{in}}(\vec{r})\right) e^{ik\phi ^{\mathrm{in}}(\vec{r})}
\end{align}
\end{subequations} 

In order to prevent any interference between these two terms in Eq.~\eqref{eq:extrafield} we deflect the desired light field by adding an additional phase ramp \cite{CURTIS2002169} equal to :
\begin{equation}
\phi _{\mathrm{disp}}(\vec{r}) = \frac{k}{f}\vec{\rho _{0}}\cdot \vec{r}
\end{equation}
to certain location.