\chapter*{Abstract}
\addcontentsline{toc}{chapter}{Abstract}
\label{ch:abstract}


\fancypagestyle{abstract_style}
{
\fancyhf{}
\lhead{{\bf Title: }\thesistitle\\
  {\bf Author: } \thesisauthor\\
  {\bf Advisor: } \thesisadvisor}
}


A beam of light is traveling electromagnetic wave and that has the ability to transfer linear and angular momentum to an illuminated object. This ability can be used to capture and guide small particles along chosen trajectories. One special scenario involves trapping and pulling objects towards the source of the beam of light over a long range, opposite to the direction of propagation. A wave that pulls has long been known in science fiction literature as a ``tractor beam''. Quite remarkably light waves that act as tractor beam have been demonstrated experimentally. To generate such a force field the mode of light, which is known as a ``tractor beam'', is a superposition of non-diffracting Bessel beams with special chosen characteristics. Generating such a mode of light involves designing the electromagnetic wave that exerts the desired force field by creating an optical system to project that mode. This thesis addresses both of these challenges and explores the nature of the ``accelerating'' mode of light that act as a tractor beam. 

The goal of this thesis is to achieve long range optical micromanipulation of colloidal particles. After a brief description to the field in Chapter \ref{ch:intro}, this thesis presents in Chapter \ref{ch:fundamentals_of_light} the formalism used to describe the propagation and diffraction of electromagnetic waves, so called topological modes of light. \ref{ch:topological_modes} introduces the holographic creation of propagation invariant modes of light and discusses their applications. In Chapter \ref{ch:intermediate} we introduce a new experimental technique called ``Intermediate Plane Holography'' which can extend the range of any non-diffracting mode of light. We demonstrate this technique through the first experimental realization of meter-class tractor beams. Finally, we use intermediate plane holography to create modes of light that appear, in themselves to be accelerating and therefore to violate  Ehrenfest's theorem. Chapter \ref{ch:accelerating} introduces and resolves this paradox.